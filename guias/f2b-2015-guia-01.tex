\documentclass[a4paper,12pt]{article}
\usepackage[spanish]{babel}
\hyphenation{}
\usepackage[utf8]{inputenc}
\usepackage[T1]{fontenc}
\usepackage{graphicx}
\usepackage[pdftex,colorlinks=true, pdfstartview=FitH, linkcolor=blue,
citecolor=blue, urlcolor=blue, pdfpagemode=UseOutlines, pdfauthor={H. Asorey},
pdftitle={Física II B - 2015}]{hyperref}
\usepackage[adobe-utopia]{mathdesign}

\hoffset -1.23cm
\textwidth 16.5cm
\voffset -2.0cm
\textheight 26.0cm

%----------------------------------------------------------------
\begin{document}
\title{
{\normalsize{Universidad Nacional de Río Negro - Profesorados de Física}}\\
Física II B 2015\\MAS: Movimiento Oscilatorio Simple\\}
\author{Asorey-Cutsaimanis}
\date{Práctica grupal, formen grupos de 3 personas}
\maketitle

\section{Didáctica}

Elija uno de los ejercicios de la sección ``Problemas'' (no repetir entre los
distintos grupos) para explicarlo en el pizarrón al resto de la clase. 

\section{Para pensar (entregar todas)}

\begin{enumerate}
	\item Un objeto se mueve con MAS de amplitud A en el extremo de un resorte.
		Si la amplitud se duplica, ¿qué sucede con la distancia total que el
		objeto recorre en un periodo? ¿Qué sucede con el periodo? ¿Qué sucede
		con la rapidez máxima del objeto?
	\item Una caja de masa $M$ que contiene un piedra de masa $m<M$ se conecta
		a un resorte horizontal ideal y oscila sobre una mesa de aire sin
		fricción. El sistema es desplazado de su posición de equilibrio, y la
		caja comienza a oscilar con amplitud $A$. En uno de los ciclos, cuando
		la caja alcanza nuevamente la posición $+A$, la piedra se sale por
		arriba sin perturbar la caja. Las siguientes características del MAS,
		¿aumentarán, disminuirán o permanecerán igual en el movimiento
		subsecuente de la caja (¡justifique!)?: a) Frecuencia
		angular; b) Período; c) Amplitud; d) Energía cinética máxima de la
		caja; e) Velocidad máxima  de la caja.
	\item Un péndulo elástico vertical se monta en el interior de un ascensor.
		¿Qué sucede (justifique) con el periodo de oscilación (aumenta,
		disminuye, no cambia) cuando el elevador:
		\begin{enumerate}
			\item acelera hacia arriba con una aceleración
				$a=3.0$\,m\,s$^{-2}$?; 
			\item se mueve hacia abajo con una velocidad $v=2.0$\,m\,s$^{-1}$
				constante?;
			\item acelera hacia abajo con una aceleración $a=3.0$\,m\,s$^{-2}$?
		\end{enumerate}
	\item Supongo que el péndulo elástico de la pregunta anterior se monta en
		la estación espacial. ¿Se observará algún cambio en la frecuencia del
		mismo? ¿Y si estuviera en la Luna? Justifique.
	\item ¿En qué punto del movimiento de un péndulo simple es máxima la
		tensión en el cordón? ¿Y mínima? En cada caso, explique su
		razonamiento.
	\item En el dispositivo armado para estudiar el movimiento oscilatorio
		amortiguado, con una pesa colgada mediante un alambre de un péndulo
		elástico y sumergida en un líquido viscoso, ¿en que momento la tensión
		sobre el alambre que sostiene la pesa sumergida es máxima? ¿en qué
		momento es mínima? ¿en algún momento se anula la tensión? Justifique.
	\item Cuando se sube a una balanza de resortes en la farmacia, ¿qué tipo de
		movimiento describe la punta de la aguja antes de detenerse y marcar su
		peso? Justifique.
\end{enumerate}

\section{Problemas}

Entregue al menos cuatro de los siguientes ejercicios. Para hacer los gráficos
puede usar excel, gnuplot o el sistema que le resulte más conveniente. 

\begin{enumerate}
	\item La punta de un diapasón efectúa 440 vibraciones completas en
		$0.25$\,s. Calcule la frecuencia $f$, la frecuencia angular $\omega$ y
		el periodo de oscilación. 
	\item Se arma un péndulo vertical con un resorte con $k=50$\,N\,m$^{-1}$
		del cual cuelga una masa $m=0.102$\,kg. A $t=0$, la masa se encuentra
		en la posición $y=0.02$\,m moviéndose hacia arriba, siendo la amplitud
		del MAS igual a $0.04$\,m. Entonces: 
		\begin{enumerate}
			\item Usando consideraciones energéticas, calcule la velocidad a $t=0$. 
			\item Suponiendo una solución del tipo $y(t)=A \sin(\omega t +
				\varphi)$, verifique que esta propuesta es solución de la
				ecuación de movimiento, obtenga las expresiones que describen
				la velocidad $v(t)$ y la aceleración $a(t)$ y los valores de
				$A$, $\omega$ y $\varphi$. Luego haga los gráficos que
				describan $y(t)$, $v(t)$ y $a(t)$ para un tiempo total igual a
				tres periodos del MAS.
		\end{enumerate}
	\item Se construye un péndulo horizontal colocando un carro de masa $m$
		sobre un riel de manera que el rozamiento entre el riel y el carro
		resulten despreciables. En cada punta del carrito se fija uno de los
		extremos de un resorte, y el otro extremo se fija, de manera que ambos
		resortes estén en tensión. Suponga que las constantes elásticas de los
		resortes son $k_1$ y $k_2$, los cuales a priori podrían no ser iguales.
		Entonces:
		\begin{enumerate}
			\item Suponiendo que $k_1 = k_2$, dibuje el diagrama de cuerpo
				libre del carrito y obtenga la posición de equilibro del carro
				tomando como origen del sistema de referencia uno de los
				extremos fijos de uno de los resortes. 
			\item Sitúe un nuevo sistema de referencia en la posición de
				equilibrio del sistema, y ahora perturbe al sistema desplazando
				al sistema a la posición $x=x_0$ en el instante $t=0$.
				Determine el período de oscilación del péndulo y la velocidad
				máxima del carro y la energía mecánica total, suponiendo que el
				carro tiene una masa $m=204$\,g y $k_1=k_2=30$\,N\,m$^{-1}$.
			\item Suponga ahora que $k_1 > k_2$. Obtenga la nueva posición de
				equilibrio y la ecuación que da el periodo de oscilación del
				péndulo en función de $m$, $k_1$ y $k_2$. 
		\end{enumerate}
	\item Del péndulo del ejercicio 2, se cuelga una nueva pesa de masa
		$m=0.051$\,kg y densidad $\rho=7.8$\,g\,cm$^{-3}$ de manera que la
		misma quede sumergida en el seno de un líquido viscoso de densidad
		$\rho_l=0.8$\,g\,cm$^{-3}$ con constante de amortiguamiento $b$.
		Encuentre para este sistema el valor $b_c$ que produce un
		amortiguamiento crítico, y haga un gráfico de $y(t)$ para los
		siguientes casos: 
		\begin{enumerate}
			\item $b=b_c / 2$ para una duración total de $10$ periodos (debe
				calcular $\omega'$)
			\item $b=b_c$ para la misma duración total que el anterior
		\end{enumerate}
\end{enumerate}

\end{document}
%%%%
