\documentclass[a4paper,12pt]{article}
\usepackage[spanish]{babel}
\hyphenation{}
\usepackage[utf8]{inputenc}
\usepackage[T1]{fontenc}
\usepackage{graphicx}
\usepackage[pdftex,colorlinks=true, pdfstartview=FitH, linkcolor=blue,
citecolor=blue, urlcolor=blue, pdfpagemode=UseOutlines, pdfauthor={H. Asorey},
pdftitle={Física II B - 2015}]{hyperref}
\usepackage[adobe-utopia]{mathdesign}

\hoffset -1.23cm
\textwidth 16.5cm
\voffset -2.0cm
\textheight 25.0cm

%----------------------------------------------------------------
\begin{document}
\title{
{\normalsize{Universidad Nacional de Río Negro - Profesorados de Física}}\\
Física II B 2015\\Ondas, nada más\\}
\author{Asorey-Cutsaimanis}
\date{Práctica individual}
\maketitle

\section{Didáctica (actividad grupal, dos grupos, Jueves 08 de octubre)}

Se implementará un entorno S.O.L.E. (ambiente de aprendizaje auto-organizado,
por sus siglas en inglés, {\emph{self organizing learning environment}}) y se
debatirán las conclusiones obtenidas por cada grupo. El docente modera la
actividad. El tema será propuesto el mismo día. Es importante contar con una
computadora o teléfono capaz de conectarse a Internet.

\section{Para pensar (entregar todas)}

\begin{enumerate}
	\item Dos ondas viajan sobre la misma cuerda. ¿Es posible que ambas
		tengan a) diferentes frecuencias?, b) diferentes longitudes de onda?,
		c) diferentes velocidades?, d) diferentes amplitudes?, e) la misma
		frecuencia, pero diferentes longitudes de onda?
	\item En una cuerda sometida a una tensión de magnitud $F$, una onda tarda
		$4.00$\,s en recorrer una distancia $d$. ¿Cuál debe ser la tensión de
		la cuerda, en relación con $F$, para que la onda demore $16.00$\,s en
		recorrer la misma distancia?
	\item En una cuerda real, la amplitud de una onda viajera disminuye con el
		tiempo. ¿Qué sucede con la energía en este caso?
	\item Las cuatro cuerdas de un violín tienen la misma longitud de vibración
		$l$, la misma tensión $F$ pero distinto diámetro. En este caso, ¿las
		ondas viajan más rápido en las cuerdas gruesas o en las delgadas? ¿Por
		qué? Compare la frecuencia fundamental de vibración (es decir, la
		frecuencia de la onda que tiene $\lambda=l$) las cuerdas gruesas y
		delgadas.
	\item Dos cuerdas con diferente densidad lineal de masa $\mu_1$ y $\mu_2$
		se unen y se estiran con una tensión de magnitud $F$. Una onda viaja
		por la cuerda y pasa por la discontinuidad de masa. Indique cuáles de
		las siguientes propiedades de la onda serán iguales a ambos lados de
		la discontinuidad y cuáles cambiarán: velocidad de la onda $v$,
		frecuencia $f$, longitud de onda $\lambda$.	Justifique físicamente cada
		respuesta.
	\item ¿Podemos producir una onda estacionaria en una cuerda superponiendo
		dos ondas que viajan en direcciones opuestas con la misma frecuencia
		pero diferente amplitud? ¿Por qué? ¿Podemos producirla superponiendo
		dos ondas que viajen en direcciones opuestas con diferente frecuencia,
		pero la misma amplitud? ¿Por qué?
	\item ¿En una onda estacionaria en una cuerda, hay transferencia neta de
		energía de un extremo al otro? Justifique su respuesta.
	\item Una diferencia de una octava (ocho tonos que van del ``Do'' al
		siguiente ``Do'') en un instrumento musical corresponde a un factor 2
		en frecuencia. Es decir, si ``La'' corresponde a una frecuencia de
		$440$\,Hz, el ``La'' de la siguiente escala tendrá una frecuencia de
		$880$\,Hz. ¿En qué factor debe aumentarse la tensión en una cuerda de
		guitarra o violín para aumentar su tono una octava? ¿Y dos octavas?
		¿por qué cree que en esos instrumentos se usan cuerdas de distintos
		materiales o grosores para los distintos tonos, si con sólo cambiar la
		tensión es posible cambiar la frecuencia? Justifique todas sus
		respuestas.
	\item Si un chelista toca una cuerda de su chelo levemente en el centro de
		la misma mientras la frota normalmente con el arco, puede producir una
		nota exactamente una octava arriba de aquella para la cual se afinó la
		cuerda. ¿Cómo es posible esto?
	\item ¿Por qué los violines son pequeños y los contrabajos son grandes?
\end{enumerate}

\section{Problemas}

Para hacer los gráficos puede usar excel, gnuplot o el sistema que le resulte
más conveniente. El Jueves 22 de Octubre deberán entregar los ejercicios
marcados con (*).

\begin{enumerate}
	\item La rapidez del sonido en aire a $20^{\mathrm{o}}$C es de $344$ m/s.
		a) Calcule la longitud de onda de una onda sonora con la frecuencia que
		corresponde a la nota fa de la cuarta octava de un piano (averigüe este
		dato en Internet), y cuántos milisegundos dura cada vibración. b)
		Calcule la longitud de onda de una onda sonora una octava más alta y
		una octava más baja que la nota del punto anterior.
	\item El 26 de diciembre de 2004 ocurrió un intenso terremoto en las costas
		de Sumatra, y desencadenó un tsunami devastador. Gracias a los
		satélites, se pudo establecer que había 800 km de la cresta de una ola
		a la siguiente, y que el periodo entre una y otra fue de 1.0 hora.
		¿Cuál fue la velocidad de esas olas en m/s? ¿Comprende ahora porque el
		efecto fue devastador? ¿Podría estimar ``a la Fermi'' la energía de la
		ola?
	\item En exploración médica se utilizan ondas sonoras con frecuencias
		mayores a las que puede percibir el oído humano ($f>20$\,kHz). Si las
		ondas utilizadas tienen una velocidad de $2000$\,m/s, y la longitud de
		onda no debería ser mayor que $1$\,mm, ¿Qué frecuencia tiene la onda
		utilizada? ¿Cuál es el periodo?
	\item (*) La ecuación de una onda transversal es $y(x,y)=6\mathrm{\ mm\ } \cos
		\left ( 4\mathrm{\ mm\ } x - 10\mathrm{\,s\ }^{-1} t \right )$.
		Obtenga: a) la amplitud, b) la longitud de onda, c) la frecuencia, d)
		velocidad y dirección de propagación de la onda.
	\item Un oscilador armónico simple en el punto $x=0$ produce ondas en una
		cuerda. El oscilador opera con una frecuencia de $60$\,Hz y una
		amplitud de $2.50$\,cm. La cuerda tiene una densidad $\mu=70$\,g/m y se
		le aplica una tensión de $9.00$\,N. Entonces:
		\begin{enumerate}
			\item Determine la rapidez de la onda. 
			\item Calcule la longitud de onda.
			\item Describa la función de onda $y(x, t)$ (suponga que el
				oscilador tiene su desplazamiento máximo hacia arriba en el
				instante $t=0$).
			\item Calcule la aceleración transversal máxima de las partículas
				de la cuerda.
		\end{enumerate}
	\item (*) Una cuerda de longitud $1.50$\,m y masa $m=0.128$\,kg está pegada al
		techo por su extremo superior, mientras que el extremo inferior
		sostiene un cuerpo de masa $M$. Si usted produce un leve pulso en la
		cuerda, las ondas generadas se desplazan hacia arriba, según la
		ecuación de onda $y(x, t)= (8.50 \mathrm{\ mm}) \cos (172 \mathrm{\ m\
		}^{-1}x - 2730 \mathrm{\ s\ }^{-1} )$. Responda:
		\begin{enumerate}
			\item ¿Cuánto tiempo tarda un pulso en viajar a lo largo de toda la
				cuerda? 
			\item ¿Cuál es la masa $M$? 
			\item ¿Cuántas longitudes de onda hay en la cuerda en cualquier
				instante?
			\item ¿Qué sucede con las ondas al llegar al extremo superior de la
				cuerda?
			\item ¿Cuál es la ecuación para las ondas en dirección descendente?
		\end{enumerate}
	\item Imagine una onda viajera, con función de onda $y(x,t)=A \cos(kx -
		\omega t)$, moviéndose a lo largo de una cuerda en la dirección $+x$. Con
		estos datos, y recordando las identidades trigonométricas para el
		coseno de la suma y resta de dos ángulos, $\cos(a \pm b) = (\cos a \cos
		b) \pm (\sin a \sin b)$, deduzca la ecuación que describe la
		correspondiente onda estacionaria en la cuerda para los siguientes
		casos:
		\begin{enumerate}
			\item los extremos de la cuerda pueden moverse (visto en clase);
			\item los extremos de la cuerda son puntos fijos.
		\end{enumerate}
		En cada caso, obtenga además la amplitud máxima, la longitud de onda,
		el período, y la posición de los nodos y antinodos de la onda
		estacionaria.
	\item (*) Una cuerda uniforme de longitud $L$ y masa $m$ está sujetada por un
		extremo a un motor que gira con velocidad angular $\Omega$.
		Despreciando el efecto de la gravedad sobre la cuerda, calcule el
		tiempo que una onda transversal necesita para recorrer la longitud de
		la cuerda $L$.
	\item Un afinador de pianos estira un alambre de piano de acero con una
		tensión de $800$\,N. El alambre tiene $0.400$\,m de longitud y una masa
		de $m=3.00$\,g. 
		\begin{enumerate}
			\item Calcule la frecuencia de su modo fundamental de vibración.
			\item Determine el número del armónico más alto que podría escuchar
				una persona que capta frecuencias de hasta $10$\,kHz.
		\end{enumerate}
	\item La función de onda de una onda estacionaria es $y(x,t) =
		4.44\mathrm{\ mm\ } \sin \left ( 32.5 \mathrm{\ m}^{-1} x \right ) \sin
		\left (754 \mathrm{\ s}^{-1} t \right )$ . Para las dos ondas viajeras que
		forman esta onda estacionaria, determine la amplitud, la longitud de
		onda, la frecuencia, la velocidad de propagación y las funciones de
		onda. Con la información suministrada, ¿puede determinar de que
		armónico se trata?. Justifique.
	\item (*) Una cuerda con ambos extremos fijos está vibrando en su tercer
		armónico. Las ondas tienen una rapidez de $192$\,m/s y una frecuencia
		de $240$\,Hz. La amplitud de la onda estacionaria en un antinodo es de
		$0.400$\,cm. Entonces
		\begin{enumerate}
			\item Calcule la amplitud del movimiento de puntos de la cuerda a
				una distancia de
				\begin{enumerate}
					\item $40.0$\,cm
					\item $20.0$\,cm; y 
					\item $10.0$\,cm
				\end{enumerate}
				del extremo izquierdo de la cuerda. 
			\item para cada uno de los puntos del inciso anterior, calcule
				cuánto tiempo tarda la cuerda en ir de su desplazamiento más
				grande hacia arriba, hasta su desplazamiento más grande hacia
				abajo.
			\item Calcule la velocidad y la aceleración transversales máximas
				de la cuerda en cada uno de los puntos del inciso a).
		\end{enumerate}
	\item (*) Una escultura de aluminio sólido se cuelga de un alambre de
		acero. La frecuencia fundamental para ondas estacionarias transversales
		en el alambre es de $440$\,Hz. Luego, la escultura (no el alambre) se
		sumerge totalmente en agua. Calcule la nueva frecuencia fundamental y
		diga por qué es una buena aproximación tratar el alambre como si
		estuviera fijo en ambos extremos (obtenga las densidades de tablas).
\end{enumerate}

\end{document}
%%%%
