\documentclass[a4paper,12pt]{article}
\usepackage[spanish]{babel}
\hyphenation{}
\usepackage[utf8]{inputenc}
\usepackage[T1]{fontenc}
\usepackage{graphicx}
\usepackage[pdftex,colorlinks=true, pdfstartview=FitH, linkcolor=blue,
citecolor=blue, urlcolor=blue, pdfpagemode=UseOutlines, pdfauthor={H. Asorey},
pdftitle={Física II B - 2015}]{hyperref}
\usepackage[adobe-utopia]{mathdesign}

\hoffset -1.23cm
\textwidth 16.5cm
\voffset -2.0cm
\textheight 25.0cm

%----------------------------------------------------------------
\begin{document}
\title{
{\normalsize{Universidad Nacional de Río Negro - Profesorados de Física}}\\
Física II B 2015\\Sonido\\}
\author{Asorey-Cutsaimanis}
\date{Práctica individual\\{\bf{Entrega Jueves 26 de Noviembre}}}
\maketitle

\section{Didáctica (actividad grupal, un grupo, Martes 24 de Noviembre)}

Discuta posibles experimentos que podría introducir en el aula en
función de las experiencias y aprendizajes desarrollados a lo largo de
este curso.

\section{Para pensar (entregar todas)}

\begin{enumerate}
	\item En el libro ``El Retorno del Rey'', de J. R. R. Tolkien,
		Aragorn sabe del acercamiento de un ejercito con mucha
		anticipación al acostarse en
		el piso y acercar su oído al suelo. Explique como esto sucede y
		justifique sus apreciaciones. 
	\item Imagine un sistema compuesto por dos gases idénticos a muy
		distinta temperatura. Una onda se propaga por el primer medio
		(más frío) y es transmitida hacia el medio más caliente.
		¿Cambia la longitud de onda?  ¿y la frecuencia? ¿y la velocidad
		de propagación?
	\item Explique la causa por la cuál la voz de una persona que
		inhaló helio se torna más aguda.
	\item ¿Qué tiene una influencia más directa sobre el volúmen de una
		onda sonora: la amplitud del desplazamiento o de los cambios de
		presión?
	\item El timbre de una cuerda de guitarra es diferente si la cuerda
		es punteada en el centro de la cuerda respecto a si es punteada
		cerca de la boca (el agujero que está en la caja). ¿Por qué?
	\item Si se reduce a la mitad la amplitud de presión de una onda
		sonora ¿en qué factor disminuye su intensidad? ¿Cuántos
		decibeles menos tiene respecto a la condición inicial? ¿cuánto
		debe aumentar la amplitud de presión para que la intensidad
		aumente un factor de $30$\,dB?
	\item Un alambre tensiondo vibra en su segundo sobretono
		produciendo un sonido con longitud de onda $\lambda$. ¿Cuál
		será la nueva longitud de onda del sonido (en términos de
		$\lambda$) si se duplica la tensión?
	\item Cuál debe ser la relación en la longitud de dos tubos de
		organo de iglesia para que produzcan sonidos con la misma
		frecuencia fundamental, si uno de ellos es abierto-abierto y el
		otro es abierto-cerrado. En ese caso, ¿habrá alguna diferencia
		entre los sonidos escuchados?
	\item En los instrumentos de viento, ¿cuál es la función de las
		llaves o pistones? ¿de qué forma estos cambian la frecuencia
		del sónido producido?
	\item Se tiene dos tubos de igual longitud pero uno de ellos es
		abierto-abierto y el otro es abierto-cerrado. ¿Qué sobretono
		debe sonar en el tubo abierto-abierto para que resuene el tubo
		abierto-cerrado? ¿Y si el que suena es el abierto-cerrado, es
		posible que resuene el abierto-abierto? En caso afirmativo diga
		en que sobretono.
	\item Haga un diagrama que explique como calcularía la velocidad de
		un avión supersónico utilizando la forma del frente de choque
		que este produce. 
\end{enumerate}

\section{Problemas}

Para hacer los gráficos puede usar excel, gnuplot o el sistema que le
resulte más conveniente. Deberán entregar los ejercicios marcados con
(*). Salvo indicación contraria, la velocidad del sonido en aire es
$v=344$\,m\,s$^{-1}$.

\begin{enumerate}
	\item (*) Imagine que una onda de frecuencia $f=1000$\,Hz tiene una
		amplitud de desplazamiento de $10^{-8}$\,m. Calcule: 
		\begin{enumerate}
			\item la amplitud de presión correspondiente;
			\item la longitud de onda de esas ondas;
			\item la amplitud de desplazamiento tal que la variación de
				presión esté en el umbral de dolor;
			\item la frecuencia y la longitud de onda que debería tener
				esta onda para que, con el mismo desplazamiento de
				$10^{-8}$\,m, la amplitud de presión sea equivalente al
				umbral de dolor ($30$\,Pa);
			\item imagine que en vez de aire, la onda original se
				propaga en agua ($\eta_{\mathrm{T}=293\mathrm{\
				K}}=4.55\times10^{-10}$\,Pa$^{-1}$). En estas
				condiciones calcule:
				\begin{enumerate}
					\item la velocidad del sonido en agua;
					\item la longitud de onda;
					\item la amplitud de la onda de presión;
					\item al amplitud del desplazamiento para que la
						amplitud de presión alcance el umbral de dolor;
					\item compare los valores obtenidos con los
						anteriores para el aire. Explique las
						diferencias observadas.
				\end{enumerate}
		\end{enumerate}
\item Dos altavoces, 1 y 2, se sitúan a $5$\,m de distancia entre si, y emiten sonido de manera uniforme y omnidireccional. La salida de potencia acústica de cada parlante es: $P_1=2\times 10^{-3}$\,W, $P_2$=$7 \times 10^{-4}$\,W. Ambos vibran en fase con una frecuencia de $344$\,Hz.
	\begin{enumerate}
		\item Calcule la intensidad sonora que genera cada parlante en un punto $X$ situado a $3$\,m del parlante 1 y $2$\,m del parlante 2, medidos sobre la línea que une a ambos parlantes.
			Calcule la intensidad sonora a 
		Determine la diferencia de fase de las dos señales en un punto
		C sobre la línea que une A a B, a 3.00 m de B y 4.00 m de A
		(figura 16.46). b) Determine la intensidad y el nivel de
		intensidad de sonido en C debidos al altavoz .A si B se apaga,
		 y haga lo mismo para el altavoz B si A se apaga. c) Con ambos
		alta- voces encendidos, determine la intensidad y el nivel de
		intensidad de sonido en C.
\end{enumerate}
\end{enumerate}
\end{document}
%%%%
