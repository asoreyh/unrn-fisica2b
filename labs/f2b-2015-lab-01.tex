\documentclass[a4paper,12pt]{article}
\usepackage[spanish]{babel}
\hyphenation{}
\usepackage[utf8]{inputenc}
\usepackage[T1]{fontenc}
\usepackage{graphicx}
\usepackage[pdftex,colorlinks=true, pdfstartview=FitH, linkcolor=blue,
citecolor=blue, urlcolor=blue, pdfpagemode=UseOutlines, pdfauthor={H. Asorey},
pdftitle={Física II B - 2015}]{hyperref}
\usepackage[adobe-utopia]{mathdesign}

\hoffset -1.23cm
\textwidth 16.5cm
\voffset -2.0cm
\textheight 26.0cm

%----------------------------------------------------------------
\begin{document}
\title{
{\normalsize{Universidad Nacional de Río Negro - Profesorados de Física}}\\
Física II B \\ Lab 01: Péndulo elástico\\}
\author{Asorey-Cutsaimanis}
\date{2015}
\maketitle

\section{}

Construya un péndulo elástico vertical, consistente en una masa $m$ soportada por un resorte de constante elástica $k$. Una vez establecida la condición de equilibrio estático, se utilizará la deformación del resorte para determinar la constante $k$ (repita el procedimiento con varias masas para minimizar los errores). Luego, perturbe el sistema moviendo la masa $m$ en dirección vertical y permita a la masa oscilar libremente en torno a la posición de equilibrio para establecer un movimiento armónico vertical. Determine el periódo de osciliación del péndulo.

Repita lo anterior con 3 resortes de distintas constantes elásticas y con 3 masas distintas, y luego: 

\begin{enumerate}
	\item {\bf{Constante Elástica}}
		\begin{enumerate}
			\item   
			\item b
		\end{enumerate}
	\item {\bf{2}}
		\begin{enumerate}
			\item 
			\item b
		\end{enumerate}
\end{enumerate}
\end{document}
%%%%
